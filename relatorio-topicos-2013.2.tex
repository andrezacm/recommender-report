%1. Introdução
%incluir:
%- problema abordado (qual o problema que o software vem resolver? que usuários ele atende? quais são os casos de uso / histórias de usuário?)
%- motivação (o que motivou o grupo a desenvolver esta ferramenta?)
%- contribuição esperada (em que aspectos sua aplicação fará uma contribuição?)
%2. Conceitos: Awareness, Coordenação, Collaboration readiness etc
%- definições para cada conceito: elaborar de forma crítica sobre os conceitos vistos em aula de acordo com o entendimento de cada membro do grupo. Eu quero um parágrafo por membro do grupo sobre cada definição e finalmente a definição final que o grupo escolheu.
%3. Processo de desenvolvimento
%- descrição das etapas e tarefas realizadas por cada membro do grupo a cada semana
%- ferramentas utilizadas 
%- problemas encontrados
%4. Discussão da relação entre problemas encontrados e conceitos
%- como os problemas poderiam ser resolvidos?
%- como as definições apresentadas anteriormente se relacionam com os problemas
%- análise crítica sobre o desempenho do grupo
%- o que será feito na terceira Unidade para corrigir os problemas?
%

\documentclass{acm_proc_article-sp}
\usepackage[utf8]{inputenc}

\begin{document}

\title{A Sample {\ttlit ACM} SIG Proceedings Paper in LaTeX Format}
\subtitle{Extended Abstract}


\numberofauthors{3}

\author{
\alignauthor
A. C. Medeiros\\
       \affaddr{Universidade Federal do Rio Grande do Norte}\\
       \email{andreza.cmedeiros@gmail.com}
\alignauthor
J. B. G. F. Filho\\
       \affaddr{Universidade Federal do Rio Grande do Norte}\\
       \email{bernardogfilho@gmail.com}
\alignauthor 
R. M. Souza\\
       \affaddr{Universidade Federal do Rio Grande do Norte}\\
       \email{rafael@maisweb.org}
}

\maketitle
\begin{abstract}
This paper provides a sample of a \LaTeX\ document which conforms to
the formatting guidelines for ACM SIG Proceedings.
It complements the document \textit{Author's Guide to Preparing
ACM SIG Proceedings Using \LaTeX$2_\epsilon$\ and Bib\TeX}. This
source file has been written with the intention of being
compiled under \LaTeX$2_\epsilon$\ and BibTeX.

The developers have tried to include every imaginable sort
of ``bells and whistles", such as a subtitle, footnotes on
title, subtitle and authors, as well as in the text, and
every optional component (e.g. Acknowledgments, Additional
Authors, Appendices), not to mention examples of
equations, theorems, tables and figures.

To make best use of this sample document, run it through \LaTeX\
and BibTeX, and compare this source code with the printed
output produced by the dvi file.
\end{abstract}

\section{Introdução}
%incluir:
%- problema abordado (qual o problema que o software vem resolver? que usuários ele atende? quais são os casos de uso / histórias de usuário?)
%- motivação (o que motivou o grupo a desenvolver esta ferramenta?)
%- contribuição esperada (em que aspectos sua aplicação fará uma contribuição?)


\section{Conceitos}
%2. Conceitos: Awareness, Coordenação, Collaboration readiness etc
%- definições para cada conceito: elaborar de forma crítica sobre os conceitos vistos em aula de acordo com o entendimento de cada membro do grupo. Eu quero um parágrafo por membro do grupo sobre cada definição e finalmente a definição final que o grupo escolheu.

\section{Processos de Desenvoldimento}
%3. Processo de desenvolvimento
%- descrição das etapas e tarefas realizadas por cada membro do grupo a cada semana
%- ferramentas utilizadas 
%- problemas encontrados

\section{Discussão}
%4. Discussão da relação entre problemas encontrados e conceitos
%- como os problemas poderiam ser resolvidos?
%- como as definições apresentadas anteriormente se relacionam com os problemas
%- análise crítica sobre o desempenho do grupo
%- o que será feito na terceira Unidade para corrigir os problemas?
%

\section{Conclusão}

%
% The following two commands are all you need in the
% initial runs of your .tex file to
% produce the bibliography for the citations in your paper.
\bibliographystyle{abbrv}
\bibliography{relatorio-topicos-2013.2}  % sigproc.bib is the name of the Bibliography in this case

\balancecolumns
% That's all folks!
\end{document}
