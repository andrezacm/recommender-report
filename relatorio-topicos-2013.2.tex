%1. Introdução
%incluir:
%- problema abordado (qual o problema que o software vem resolver? que usuários ele atende? quais são os casos de uso / histórias de usuário?)
%- motivação (o que motivou o grupo a desenvolver esta ferramenta?)
%- contribuição esperada (em que aspectos sua aplicação fará uma contribuição?)
%2. Conceitos: Awareness, Coordenação, Collaboration readiness etc
%- definições para cada conceito: elaborar de forma crítica sobre os conceitos vistos em aula de acordo com o entendimento de cada membro do grupo. Eu quero um parágrafo por membro do grupo sobre cada definição e finalmente a definição final que o grupo escolheu.
%3. Processo de desenvolvimento
%- descrição das etapas e tarefas realizadas por cada membro do grupo a cada semana
%- ferramentas utilizadas 
%- problemas encontrados
%4. Discussão da relação entre problemas encontrados e conceitos
%- como os problemas poderiam ser resolvidos?
%- como as definições apresentadas anteriormente se relacionam com os problemas
%- análise crítica sobre o desempenho do grupo
%- o que será feito na terceira Unidade para corrigir os problemas?
%

\documentclass{acm_proc_article-sp}
\usepackage[utf8]{inputenc}

\begin{document}

\title{Tópicos Especiais em Engenharia de Software III}
\subtitle{Relatório Final}


\numberofauthors{3}
  
\author{
\alignauthor
A. C. Medeiros\\
       \affaddr{Universidade Federal do Rio Grande do Norte}\\
       \email{andreza.cmedeiros@gmail.com}
\alignauthor
J. B. G. F. Filho\\
       \affaddr{Universidade Federal do Rio Grande do Norte}\\
       \email{bernardogfilho@gmail.com}
\alignauthor 
R. M. Souza\\
       \affaddr{Universidade Federal do Rio Grande do Norte}\\
       \email{rafael@maisweb.org}
}

\maketitle
\begin{abstract}
O objetivo deste documento é relatar de forma analítica o processo colaborativo
desenvolvido pela nossa equipe ao longo de um período de desenvolvimento 
da ferramenta Code Recommender. Que problemas encontramos? Como estes problemas nos 
afetaram? Quais problemas solucionamos? Como solucionamos? Vamos tentar relatar a 
nossa experiência através dos conceitos abordados em sala de aula em torno de 
sistemas colaborativos.
\end{abstract}

\section{Introdução}
Quando temos uma nova necessidade no meio da construção de algum software,
geralmente é uma boa idéia usar algo que já está pronto,
usado e testado por outras pessoas que tiveram uma necessidade semelhante anteriormente.
Um caso comum é recorremos a bibliotecas open source que estejam disponíveis na rede (github.com, rubygems, etc.)
que normalmente possuem várias alternativas prontas para as necessidades menos incomuns.

Um problema nesse cenário é que, com a rápida evolução de bibliotecas existentes e o rápido surgimento de novas alternativas,
não é uma tarefa simples escolher qual das alternativas disponíveis atende melhor as particularidades especiais de um projeto.
Muitas vezes uma biblioteca não está sendo bem mantida, ou é uma biblioteca legada com alternativas claramente melhores. Existem
vários motivos para querer usar uma biblioteca específica no lugar de outras.

O propósito do Recommender é facilitar essa decisão, inicialmente mostrando os atributos de cada biblioteca e facilitando a comparação entre várias bibliotecas.

Com isso, esperamos que desenvolvedores possam constatar facilmente qual a melhor
biblioteca para seus propósitos.
% ps. tentei colocar os 3 pontos sem necessariamente dizer as palavras "nossa motivação...", "espero que o Recommender contribuar...".
%incluir:
%- problema abordado (qual o problema que o software vem resolver? que usuários ele atende? quais são os casos de uso / histórias de usuário?)
%- motivação (o que motivou o grupo a desenvolver esta ferramenta?)
%- contribuição esperada (em que aspectos sua aplicação fará uma contribuição?)


\section{Conceitos}
%2. Conceitos: Awareness, Coordenação, Collaboration readiness etc
%- definições para cada conceito: elaborar de forma crítica sobre os conceitos vistos em aula de acordo com o entendimento de cada membro do grupo. Eu quero um parágrafo por membro do grupo sobre cada definição e finalmente a definição final que o grupo escolheu.

% Common ground, coupling of work, collaboration readiness, and collaboration technology readiness.
%Common ground,
%• Coupling (dependencies) of group work,
%• Collaboration Readiness, the motivation for co-workers to collaborate, and
%• Collaboration Technology Readiness, the current level of groupware assimilated by the team.

Este relatório tem o objetivo de analisar de forma crítica o esforço colocado na tentativa de alcançar uma comunicação e colaboração de qualidade na nossa equipe durante o desenvolvimento do projeto Code Recommender. Mas antes, é preciso entender os conceitos abordados em sala de aula, pois só assim podemos desenvolver uma análise rica e coerente.

Dentre os conceitos abordados em sala de aula, destacaremos os seguintes:

\begin{enumerate}
\item \textsl{Awareness}
\item \textsl{Common Ground}
\item \textsl{Coupling of Work}
\item \textsl{Collaboration Readiness}
\item \textsl{Technology Readiness}
\end{enumerate}
Neste relatório, cada membro da equipe colaborará com uma definição de cada conceito, do modo como ele entendeu. O objetivo disto é enriquecer a conclusão final de cada conceito em questão.

\subsection{Awareness}
Segue abaixo as definições de Awareness descritas por cada integrante do grupo.

\begin{quote}\textbf{Bernardo Gurgel}
Quando estamos trabalhando dentro de um sistema colaborativo, \textsl{awareness} se diz respeito ao quanto a informação está disponível e visível para os membros envolvidos. Garantir uma melhor \textsl{awareness} é tornar a atividade de um membro visível aos outros envolvidos naquele projeto.
\end{quote}

\begin{quote}\textbf{Andreza Medeiros}

\end{quote}

\begin{quote}\textbf{Rafael Morais}
\end{quote}

\subsection{Common Ground}
Segue abaixo as definições de Common Ground descritas por cada integrante do grupo.

\begin{quote}\textbf{Bernardo Gurgel}
\textsl{Common Ground} se refere a toda informação compartilhada entre dois ou mais envolvidos no sistema colaborativo. Para termos um cenário de common ground, o fato desta informação ser compartilhada deve ser de conhecimento claro dos envolvidos (ex: eu sei que ele sabe que eu sei).
\end{quote}

\begin{quote}\textbf{Andreza Medeiros}
Commom Ground é uma characteristica dos participantes de um processo de colaboração (citar o distance metters). Essa característica conciste em que estes partipantes tem um conhecimento em comum, ou de um mesmo contexto. Este conhecimento pode estar relacionado a cultura, conhecimento técnico ou, até mesmo, do local onde está situado. 
\end{quote}

\begin{quote}\textbf{Rafael Morais}
\end{quote}

\subsection{Coupling of Work}
Segue abaixo as definições de Coupling of Work descritas por cada integrante do grupo.

\begin{quote}\textbf{Bernardo Gurgel}
\end{quote}

\begin{quote}\textbf{Andreza Medeiros}
Coupling Work é o conceito de que um trabalho está acoplado a outro, ou seja, o quanto uma tarefa depende de outra para ser realizada. 
\end{quote}

\begin{quote}\textbf{Rafael Morais}
\end{quote}

\subsection{Collaboration Readiness}
Segue abaixo as definições de Collaboration Readiness descritas por cada integrante do grupo.

\begin{quote}\textbf{Bernardo Gurgel}
\textsl{Collaboration Readiness} é o quanto um ou mais envolvidos se encontram prontos, aptos a colaborar com uma determinada atividade dentro de um sistema colaborativo.
\end{quote}

\begin{quote}\textbf{Andreza Medeiros}
Collaboration Readiness se refere ao quanto os participantes de uma atividade colaborativa estão dispostos a colaborar.
\end{quote}

\begin{quote}\textbf{Rafael Morais}
\end{quote}

\subsection{Technology Readiness}
Segue abaixo as definições de Technology Readiness descritas por cada integrante do grupo.

\begin{quote}\textbf{Bernardo Gurgel}
\textsl{Technology Readiness} se trata do quanto um ou mais envolvidos no sistema colaborativo estão preparados para utilizar uma determinada tecnologia. Neste contexto entra diversas variáveis, como conhecimentos requisitados pela tecnologia, maturidade tanto do usuário quanto da tecnologia, tempo que o usuário pode se dedicar para o uso da tecnologia, etc.
\end{quote}

\begin{quote}\textbf{Andreza Medeiros}
Techology Readiness está associada ao quanto as pessoas estão preparadas para adotar uma ferramenta para colaborarem. A dificuldade de adoção de uma nova tecnologia para colaboração pode estar relacionada a falta de estrutura fisica, falta de conhecimento técnico ou, até mesmo, a dificuldade no desenvolvimento do hábito de utilizar a ferramenta.
\end{quote}

\begin{quote}\textbf{Rafael Morais}
\end{quote}

\section{Processo de Desenvolvimento}
\subsection{Tarefas Realizadas}
A tabela a seguir mostra as tarefas realizadas por semana e os membros do grupo que participaram.
Nela não estão incluídos os outros membros do projeto (Waldyr, Jonathan, Luís e Fernando).

\begin{table}[h]
  \caption{Tarefas por semana}
  \begin{tabular}{|cp{3cm}p{3cm}|}
    \hline
    Semana&Tarefa&Pessoas\\
    \hline
    29/09 a 05/10&Bernardo e Andreza&Discussão de features e possíveis implementações\\
    06/10 a 12/10&Bernardo, Andreza e Rafael&Organização do time, metas e novas discussões\\
    13/10 a 19/10&Bernardo, Andreza e Rafael&Protótipo de tela\\
    20/10 a 26/10&Bernardo, Andreza e Rafael&Protótipo de tela e inclusão de protótipos no projeto Rails\\
    \hline
  \end{tabular}
\end{table}

O projeto está andando claramente devagar, com pouca coisa pronta: 2 protótipos de tela da parte do front-end e,
noback-end, os scripts que irão alimentar a base de dados em um primeiro momento.

O nosso próximo passo é transformar os protótipos (com dados fictícios) em telas que puxam dados reais.
Para isso, precisamos que os modelos de dados estejam criados e a base de dados alimentada com os dados iniciais.

\subsection{Ferramentas Utilizadas}


\subsection{Problemas Encontrados}

\section{Discussão}
%4. Discussão da relação entre problemas encontrados e conceitos
%- como os problemas poderiam ser resolvidos?
%- como as definições apresentadas anteriormente se relacionam com os problemas
%- análise crítica sobre o desempenho do grupo
%- o que será feito na terceira Unidade para corrigir os problemas?
%

\section{Conclusão}

%
% The following two commands are all you need in the
% initial runs of your .tex file to
% produce the bibliography for the citations in your paper.
\bibliographystyle{abbrv}
\bibliography{relatorio-topicos-2013.2}  % sigproc.bib is the name of the Bibliography in this case

\balancecolumns
% That's all folks!
\end{document}
