%1. Introdução
%incluir:
%- problema abordado (qual o problema que o software vem resolver? que usuários ele atende? quais são os casos de uso / histórias de usuário?)
%- motivação (o que motivou o grupo a desenvolver esta ferramenta?)
%- contribuição esperada (em que aspectos sua aplicação fará uma contribuição?)
%2. Conceitos: Awareness, Coordenação, Collaboration readiness etc
%- definições para cada conceito: elaborar de forma crítica sobre os conceitos vistos em aula de acordo com o entendimento de cada membro do grupo. Eu quero um parágrafo por membro do grupo sobre cada definição e finalmente a definição final que o grupo escolheu.
%3. Processo de desenvolvimento
%- descrição das etapas e tarefas realizadas por cada membro do grupo a cada semana
%- ferramentas utilizadas 
%- problemas encontrados
%4. Discussão da relação entre problemas encontrados e conceitos
%- como os problemas poderiam ser resolvidos?
%- como as definições apresentadas anteriormente se relacionam com os problemas
%- análise crítica sobre o desempenho do grupo
%- o que será feito na terceira Unidade para corrigir os problemas?
%

\documentclass{acm_proc_article-sp}
\usepackage[utf8]{inputenc}

\begin{document}

\title{Tópicos Especiais em Engenharia de Software III}
\subtitle{Relatório Final}


\numberofauthors{3}
  
\author{
\alignauthor
J. B. G. F. Filho\\
       \affaddr{Universidade Federal do Rio Grande do Norte}\\
       \email{bernardogfilho@gmail.com}
\alignauthor 
R. M. Souza\\
       \affaddr{Universidade Federal do Rio Grande do Norte}\\
       \email{rafael@maisweb.org}
\alignauthor
A. C. Medeiros\\
       \affaddr{Universidade Federal do Rio Grande do Norte}\\
       \email{andreza.cmedeiros@gmail.com}
}

\maketitle
\begin{abstract}
O objetivo deste documento é relatar de forma analítica o processo colaborativo
desenvolvido pela nossa equipe ao longo de um período de desenvolvimento 
da ferramenta Code Recommender. Que problemas encontramos? Como estes problemas nos 
afetaram? Quais problemas solucionamos? Como solucionamos? Vamos tentar relatar a 
nossa experiência através dos conceitos abordados em sala de aula em torno de 
sistemas colaborativos.
\end{abstract}

\section{Introdução}
Quando temos uma nova necessidade no meio da construção de algum software,
geralmente é uma boa idéia usar algo que já está pronto,
usado e testado por outras pessoas que tiveram uma necessidade semelhante anteriormente.
Um caso comum é recorremos a bibliotecas open source que estejam disponíveis na rede (github.com, rubygems, etc.)
que normalmente possuem várias alternativas prontas para as necessidades menos incomuns.

Um problema nesse cenário é que, com a rápida evolução de bibliotecas existentes e o rápido surgimento de novas alternativas,
não é uma tarefa simples escolher qual das alternativas disponíveis atende melhor as particularidades especiais de um projeto.
Muitas vezes uma biblioteca não está sendo bem mantida, ou é uma biblioteca legada com alternativas claramente melhores. Existem
vários motivos para querer usar uma biblioteca específica no lugar de outras.

O propósito do Recommender é facilitar essa decisão, inicialmente mostrando os atributos de cada biblioteca e facilitando a comparação entre várias bibliotecas.

Com isso, esperamos que desenvolvedores possam constatar facilmente qual a melhor
biblioteca para seus propósitos.
% ps. tentei colocar os 3 pontos sem necessariamente dizer as palavras "nossa motivação...", "espero que o Recommender contribuar...".
%incluir:
%- problema abordado (qual o problema que o software vem resolver? que usuários ele atende? quais são os casos de uso / histórias de usuário?)
%- motivação (o que motivou o grupo a desenvolver esta ferramenta?)
%- contribuição esperada (em que aspectos sua aplicação fará uma contribuição?)


\section{Conceitos}
%2. Conceitos: Awareness, Coordenação, Collaboration readiness etc
%- definições para cada conceito: elaborar de forma crítica sobre os conceitos vistos em aula de acordo com o entendimento de cada membro do grupo. Eu quero um parágrafo por membro do grupo sobre cada definição e finalmente a definição final que o grupo escolheu.

% Common ground, coupling of work, collaboration readiness, and collaboration technology readiness.
%Common ground,
%• Coupling (dependencies) of group work,
%• Collaboration Readiness, the motivation for co-workers to collaborate, and
%• Collaboration Technology Readiness, the current level of groupware assimilated by the team.

Este relatório tem o objetivo de analisar de forma crítica o esforço colocado na tentativa de alcançar uma comunicação e colaboração de qualidade na nossa equipe durante o desenvolvimento do projeto Code Recommender. Mas antes, é preciso entender os conceitos abordados em sala de aula, pois só assim podemos desenvolver uma análise rica e coerente.

Dentre os conceitos abordados em sala de aula, destacaremos os seguintes:

\begin{enumerate}
\item \textsl{Awareness}
\item \textsl{Common Ground}
\item \textsl{Coupling of Work}
\item \textsl{Collaboration Readiness}
\item \textsl{Technology Readiness}
\item \textsl{Awareness Network}
\end{enumerate}
Neste relatório, cada membro da equipe colaborará com uma definição de cada conceito, do modo como ele entendeu. O objetivo disto é enriquecer a conclusão final de cada conceito em questão.

\subsection{Awareness}
Segue abaixo as definições de Awareness descritas por cada integrante do grupo.

\begin{quote}\textbf{Bernardo Gurgel}
Quando estamos trabalhando dentro de um sistema colaborativo, \textsl{awareness} se diz
respeito ao quanto a informação está disponível e visível para os membros envolvidos.
Garantir uma melhor \textsl{awareness} é tornar a atividade de um membro visível aos outros envolvidos naquele projeto.
\end{quote}

\begin{quote}\textbf{Andreza Medeiros}
\textsl{Awareness} pode ser entendido como a consciência ou a percepção que as pessoas envolvidas em um processo de colaboração tem dos fatos. Estes fatos podem estar relacionados a objetivos que as pessoas tem em comum ao realizar uma tarefa.
\end{quote}

\begin{quote}\textbf{Rafael Morais}
  \textsl{Awareness} é a consciência dos membros do projeto em relação à coisas importantes do projeto.
  Essas coisas importantes incluem, mas não estão limitados a:
  \begin{itemize}
    \item O trabalho dos outros membros, o que eles estão fazendo, o que estão precisando
    \item As necessidades e problemas atuais do projeto em si. Por exemplo, um certo problema
      que ninguem ainda conseguiu propor uma solução aceitável, deadlines, etc.
    \item Os atributos necessários pra fazer seu trabalho, como documentos de requisitos, entrevistas, etc.
    \item Tarefas disponíveis, tarefas terminadas, tarefas problemáticas
  \end{itemize}
\end{quote}

\subsection{Common Ground}
Segue abaixo as definições de Common Ground descritas por cada integrante do grupo.

\begin{quote}\textbf{Bernardo Gurgel}
  \textsl{Common Ground} se refere a toda informação compartilhada entre dois ou mais envolvidos no sistema colaborativo.
  Para termos um cenário de common ground, o fato desta informação ser compartilhada deve ser de conhecimento
  claro dos envolvidos (ex: eu sei que ele sabe que eu sei).
\end{quote}

\begin{quote}\textbf{Andreza Medeiros}
\textsl{Commom Ground} é uma característica dos participantes de um processo de colaboração\cite{Olson:DM}. Essa característica consiste em que estes participantes tem um conhecimento em comum, ou de um mesmo contexto. Este conhecimento pode estar relacionado a cultura, conhecimento técnico ou, até mesmo, do local onde está situado. 
\end{quote}

\begin{quote}\textbf{Rafael Morais}
  \textsl{Common Ground} é uma expressão comum no inglês e acredito que o significado no nosso contexto foge pouco (se foge) do comum.
  É toda a bagagem cultural e social que um grupo tem em comum.
  Por exemplo, crescemos no mesmo país, escutamos o mesmo tipo de música, temos o mesmo gosto por algoritmos experimentais.

  Acredito que há uma confusão como se houvesse uma interseção entre \textsl{awareness} e \textsl{common ground}. Acredito que a
  \textsl{awareness} tem a ver apenas com o projeto e é sobre você ter uma informação compartilhada. \textsl{Common ground} tem muito
  mais a ver com o que você sabe da vida que do projeto.
\end{quote}

\begin{quote}\textbf{Definição do Grupo}
\textsl{Commom Ground} está relacionado ao conhecimento compartilhado de um grupo em um determinado contexto. Este conhecimento em comum está associado a bagagem cultural e social que um grupo tem em comum. 
\end{quote}

\subsection{Coupling of Work}
Segue abaixo as definições de \textsl{Coupling of Work} descritas por cada integrante do grupo.

\begin{quote}\textbf{Bernardo Gurgel}
Encontramos uma situação de coupling of work ao depararmos com uma atividade que depende da atividade de outra pessoa para ser completada, ou progredida.
\end{quote}

\begin{quote}\textbf{Andreza Medeiros}
\textsl{Coupling of Work} é o conceito de que um trabalho está acoplado a outro, ou seja, o quanto uma tarefa depende de outra para ser realizada. 
\end{quote}

\begin{quote}\textbf{Rafael Morais}
  \textsl{Coupling of Work} é o acoplamento entre tarefas, de modo que uma tarefa depende da outra. O conceito é levado em conta dentro de outros
  conceitos, como \textsl{awareness network} que foca em compartilhamento exatamente de tarefas que estão sendo desenvolvidas para que não haja
  tarefas acopladas simultâneas.
\end{quote}

\begin{quote}\textbf{Definição do Grupo}
\textsl{Coupling of Work} é o acoplamento entre tarefas, de modo que uma tarefa depende de outra para ser realizada. No contexto de um trabalho trabalho distribuído em que pessoas dependem do resultado de outras pessoas, podemos relacionar este conceito com o de \textsl{awareness network} que foca no compartilhamento suas ações e resultados de suas atividades.
\end{quote} 

\subsection{Collaboration Readiness}
Segue abaixo as definições de Collaboration Readiness descritas por cada integrante do grupo.

\begin{quote}\textbf{Bernardo Gurgel}
\textsl{Collaboration Readiness} é o quanto um ou mais envolvidos se encontram prontos, aptos a colaborar com uma determinada atividade dentro de um sistema colaborativo.
\end{quote}

\begin{quote}\textbf{Andreza Medeiros}
  \textsl{Collaboration Readiness} se refere ao quanto os participantes de uma atividade colaborativa estão dispostos a colaborar.
\end{quote}

\begin{quote}\textbf{Rafael Morais}
  \textsl{Collaboration Readiness} é a motivação dos colaboradores cooperarem.
\end{quote}

\begin{quote}\textbf{Bernardo Gurgel} se refere a motivação, disposição e aptidão dos membros de um grupo colaborarem entre si.
\textsl{Collaboration Readiness} 
\end{quote}

\subsection{Technology Readiness}
Segue abaixo as definições de Technology Readiness descritas por cada integrante do grupo.

\begin{quote}\textbf{Bernardo Gurgel}
\textsl{Technology Readiness} se trata do quanto um ou mais envolvidos no sistema colaborativo estão preparados para utilizar uma determinada tecnologia. Neste contexto entra diversas variáveis, como conhecimentos requisitados pela tecnologia, maturidade tanto do usuário quanto da tecnologia, tempo que o usuário pode se dedicar para o uso da tecnologia, etc.
\end{quote}

\begin{quote}\textbf{Andreza Medeiros}
\textsl{Technology Readiness} está associada ao quanto as pessoas estão preparadas para adotar uma ferramenta para colaborarem. A dificuldade de adoção de uma nova tecnologia para colaboração pode estar relacionada a falta de estrutura física, falta de conhecimento técnico ou, até mesmo, a dificuldade no desenvolvimento do hábito de utilizar a ferramenta.
\end{quote}

\begin{quote}\textbf{Rafael Morais}
  \textsl{Technology Readiness} mede o quanto um grupo em uma organização está preparado para adotar uma nova tecnologia. Os problemas podem ser
  relacionados ao conhecimento do ambiente da tecnologia (ou ter o \textsl{common ground} necessário) ou a ter os requisitos necessários daquela
  tecnologia.
  O common ground da tecnologia, é, por exemplo, saber usar e se comunicar eficientemente através de email para usar gmail, outlook, etc.
  Os requerimentos da tecnologia podem ir desde possuir o hardware necessário, quanto a quantidade certa de pessoas ou outras ferramentas necessárias.
\end{quote}

\subsection{Awareness Network}
Segue abaixo as definições de \textsl{Awareness Network} descritas por cada integrante do grupo.

\begin{quote}\textbf{Bernardo Gurgel}
\textsl{Awareness Network} se refere ao sistema de distruição de informação para que se obtenha uma boa de \textsl{awareness information}.
\end{quote}

\begin{quote}\textbf{Andreza Medeiros}
O conceito de \textsl{Awareness Network} está diretamente relacionado com o conceito de \textsl{awareness} no sentido de que os membros para ter uma percepção dos fatos num processo de colaboração compartilham suas ações e resultados com outros participantes. De modo que um participante possa monitorar as ações de outro participante e que este também seja monitorado. 
\end{quote}

\begin{quote}\textbf{Rafael Morais}
  O conceito de \textsl{Awareness Network} é relacionado com as ações realizadas pelos membros de um projeto no sentido de
  monitorar as ações dos outros membros, para que ele possa saber o que vai afetar as próprias ações, e as ações realizadas no sentido de
  mostrar as suas ações para os outros membros de tal modo que eles possam monitorar facilmente.
\end{quote}

\begin{quote}\textbf{Definição do Grupo}
\textsl{awareness network}
\end{quote}
 
\section{Processo de Desenvolvimento}
\subsection{Tarefas Realizadas}
A tabela a seguir mostra as tarefas realizadas por semana e os membros do grupo que participaram.
Nela não estão incluídos os outros membros do projeto (Waldyr, Jonathan, Luís e Fernando).

\begin{table}[h]
  \caption{Tarefas por semana}
  \begin{tabular}{|cp{3cm}p{3cm}|}
    \hline
    Semana&Tarefa&Pessoas\\
    \hline
    29/09 a 05/10&Bernardo e Andreza&Discussão de features e possíveis implementações\\
    06/10 a 12/10&Bernardo, Andreza e Rafael&Organização do time, metas e novas discussões\\
    13/10 a 19/10&Bernardo, Andreza e Rafael&Protótipo de tela\\
    20/10 a 26/10&Bernardo, Andreza e Rafael&Protótipo de tela e inclusão de protótipos no projeto Rails\\
    \hline
  \end{tabular}
\end{table}

O projeto está andando claramente devagar, com pouca coisa pronta: 2 protótipos de tela da parte do front-end e,
noback-end, os scripts que irão alimentar a base de dados em um primeiro momento.

O nosso próximo passo é transformar os protótipos (com dados fictícios) em telas que puxam dados reais.
Para isso, precisamos que os modelos de dados estejam criados e a base de dados alimentada com os dados iniciais.

\subsection{Ferramentas Utilizadas}

Desde o começo do projeto, até antes que qualquer coisa tenha sido realmente feita, começamos a usar ferramentas para possibilitar
a comunicação do grupo. Algumas foram abandonadas, outras mantidas e outras adotadas posteriormente.

Inicialmente, iríamos começar com as seguintes:

\begin{itemize}
  \item Github (repositório)
  \item Gmail
  \item Google Drive
  \item HipChat (chat)
  \item Pivotal Tracker (issue tracker)
  \item Facebook Messenger (chat)
\end{itemize}

Acabamos por abandonar o HipChat, Facebook Messenger e Pivotal Tracker, e aderimos ao Google Hangout e Google Groups.

No final, ficamos com a seguinte configuração:

\begin{itemize}
  \item Github (repositório)
  \item Github (issue tracker)
  \item Gmail
  \item Google Groups
  \item Google Drive
  \item Google Hangout (chat, voz, compartilhamento de tela)
\end{itemize}

\subsection{Problemas Encontrados}
\subsubsection{Pivotal Tracker}
O Pivotal Tracker foi utilizado no início para tentar coordenar o projeto, escrever hitórias de usuários e atribuir atividades.

Tivemos problemas com relação a falta de utilização do Pivotal Tracker e acreditamos que
as causas incluem ele ser um software a mais e que parte dos membros do projeto não estavam familiarizados
com a sua utilização.

\subsubsection{Facebook Messenger}
O chat do facebook tem sido utilizado para conversas rápidas entre alguns membros pela facilidade e familiaridade com a ferramenta.

O maior problema era que o chat era utilizado com poucos presentes, e depois, os outros não tinham conhecimento
que algo tinha sido discutido.

\subsubsection{Gmail}
Gmail está dentre uma das principais ferramentas que utilizamos e está sendo o recurso primário de comunicação.
Se trata apenas de uma aplicação de e-mail da Google, pela qual marcamos reuniões e muitas vezes discutimos alguns assuntos.

O maior problema na utilização do Gmail é quando alguem manda um email abrindo uma discussão e os outros não dão continuidade.
Isso parece estar relacionado com o ambiente do email em si, onde nem sempre parece que há uma responsabilidade em responder um email.

\subsubsection{Google Drive}
Google Drive é um aplicativo de compartilhamento de arquivos e documentos na nuvem. Usamos ele para documentar requisitos e outras ideias gerais do projeto.

O Google Drive não apresentou problemas na sua utilização principal, que é colaborar em um documento em tempo real.
O problema que enfrentamos foi ao tentar discutir sobre os assuntos do documento no próprio google drive, pois ele não
notifica os membros que está havendo uma uma discussão ou que comentários foram colocados no texto.

\subsubsection{Hipchat}
O Hipchat é uma ferramenta de bate-papo persistente.

Nós começamos a usa-lo no início do projeto, mas não se mostrou muito efetivo
para comunicação do grupo todo. Algumas pessoas aderiram ao seu uso e outras não. Isso trouxe alguns problemas.
Certa vez, marcamos uma reunião pelo Hipchat e um dos integrantes não compareceu porque não checou o hipchat, e ele não notifica por email
sem uma citação.

\section{Discussão}
%4. Discussão da relação entre problemas encontrados e conceitos
%- como os problemas poderiam ser resolvidos?
%- como as definições apresentadas anteriormente se relacionam com os problemas
%- análise crítica sobre o desempenho do grupo
%- o que será feito na terceira Unidade para corrigir os problemas?
%
\subsection{Geral}
Acreditamos que o maior problema para o andamento do trabalho esteja de verdade na \textbf{vontade de fazer}
\subsection{Facebook Messenger}
O Facebook Messenger não contribuía para um boa \textslt{awareness}. Como já relatado, 
o problema do uso desta ferramenta é que nem sempre todos os integrantes do grupo estavam
presentes no seu chat. Nem todos ficavam conscientes do que havia-se debatido. Investindo mais em
uma \textslt{awareness network}, mostrando sobre o que está sendo discutido e o que está sendo feito via email (ou lista de emails)
pretendemos contornar esse problema.

\subsection{Gmail}
O problema que relatamos ao usarmos o Gmail, não chega bem a ser um problema, mas sim uma forma
de avaliarmos o quanto alguém está disposto a colaborar, participar das conversas. Através
do Gmail, pudemos identificar que nós poderíamos estar lidando com um problema de 
\textslt{Collaboration Readiness} em nosso sistema colaborativo. Fizemos algumas discussões, 
importantes inclusive, através do Gmail. Mas nem todos participavam. Estávamos conscientes que 
ele estava sendo lido pelas pessoas, porque sabíamos que o uso de Gmail era de costume entre todos
os envolvidos.

\subsection{Google Drive}
O problema do Google Drive está relacionado com a \textslt{Awareness Information}. Como a ferramenta
não notifica os seus usuários sobre mudanças no documento, não havia como saber se alguém alterou
um requisito, ou adicionou um wireframming, etc. Sentimos falta de um \textslt{broadcasting} de 
informação para que os envolvidos ficassem conscientes de mudanças.




\section{Conclusão}
Ao longo do processo de desenvolvimento do Code Recommender, desde a etapa de levantamento de 
requisitos ao desenvolvimento de protótipos e implementação de códigos, encontramos diversas 
dificuldades e problemas em nosso sistema colaborativo. Encontramos algumas soluções 
com a utilização de ferramentas, com isso conseguimos melhorar alguns pontos que estavam frágeis
em nosso processo colaborativo. Porém, ferramentas nem sempre são suficientes. Foi preciso trabalhar
e se adequar a uma cultura de trabalho, que em nosso caso, foi sendo desenvolvida lentamente (e ainda está
em processo de desenvolvimento). 

Apesar de termos fortalecido alguns pontos em nosso sistema colaborativo, sentimos que ainda estamos longe de alcançar
um ambiente ideal para colaboração. Talvez, este ideal seja impossível de ser alcançado. Ao lidar com um ambiente de colaboração,
você lida com uma série de variáveis que refletem em incontáveis perfis de combinações. Encontrar a melhor maneira possível de 
lidar com estas variáveis é um processo muito delicado, porém muito importante. Um software é geralmente desenvolvido por diversos desenvolvedores, designers e outros tipos de profissionais. Portanto, colaboração é uma peça fundamental no processo de desenvolvimento
de um software. É um fator que influi diretamnete na qualidade final do produto.




%
% The following two commands are all you need in the
% initial runs of your .tex file to
% produce the bibliography for the citations in your paper.
\bibliographystyle{abbrv}
\bibliography{relatorio-topicos-2013.2}  % sigproc.bib is the name of the Bibliography in this case

\balancecolumns
% That's all folks!
\end{document}
